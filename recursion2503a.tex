% recursion2503a
% https://docs.google.com/document/d/1Y3icfkrZiCzPpfWM5InAKiqntH-Sqzrmy84j1pS1csM/edit?tab=t.0
% !TEX encoding = UTF-8 Unicode
% https://www.site24x7.com/tools/time-stamp-converter.html 1740929447
% https://www.google.com/search?q=current+date+and+time 10:31 a.m. Sunday, March 2, 2025 (EST) Time in Montreal, QC
\documentclass[12pt,letterpaper]{article}
\usepackage{fancyhdr}
\fancyhf{}
\fancyfoot[R]{{\tiny recursion2503a,\ \filemodprintdate{\jobname},\ \filemodprinttime{\jobname},\ 1740929447}} % https://www.site24x7.com/tools/time-stamp-converter.html (https://www.unixtimestamp.com/ bad)
\renewcommand{\headrulewidth}{0pt}
\fancyfoot[C]{\thepage}
\usepackage[T1]{fontenc}
\usepackage[utf8]{inputenc}
\usepackage{amssymb,amsmath,amsthm} 
%\usepackage[letterpaper,top=30pt,left=40pt,right=40pt,bottom=60pt]{geometry}
\usepackage[letterpaper,top=50pt,left=60pt,right=60pt,bottom=80pt]{geometry}
\usepackage{filemod}%https://stackoverflow.com/questions/2118972/latex-command-for-last-modified
\usepackage{microtype}
%\usepackage{tikz-cd}
%\usepackage{varwidth}
%\usepackage{comment}
\usepackage[pdfusetitle]{hyperref}
%\usepackage{marvosym}% emoji https://tex.stackexchange.com/questions/3695/smileys-in-latex https://ctan.org/pkg/marvosym?lang=en
\pagestyle{fancy}
%\pagestyle{empty}
%\setlength\parindent{0pt}
\setlength{\parskip}{5pt} % variable
%\renewcommand{\baselinestretch}{1.1} % variable
\newcommand{\nn}{\noindent}
\begin{document}% \tiny \scriptsize \footnotesize \small \normalsize \large \Large \LARGE \huge \Huge 
\begin{center}
{\Huge Transfinite Recursion}\bigskip 

Pierre-Yves Gaillard
\end{center}

\nn The only purpose of this short text\footnote{available at \url{https://github.com/Pierre-Yves-Gaillard/The-Transfinite-Recursion-Theorem}} is to point out that the Transfinite Recursion Theorem is a particular case of the Transfinite Induction Theorem. 

For any member $x$ of any ordered set $X$, we denote by $X_x$ the set of all those $y$ which are less than $x$, that is $X_x=\{y\in X\ |\ y<x\}$. 

Say that a subset $S$ of an ordered set $X$ is \textbf{inductive} if, for all $x$ in $X$, the condition $X_x\subset S$ implies $x\in S$. 

\nn\textbf{Transfinite Induction Theorem.} Let $S$ be a subset of a well-ordered set $X$. Then $S=X$ if and only if $S$ is inductive. 

The proof is well-known and easy. 

\nn\textbf{Transfinite Recursion Theorem.} Let $X$ be well-ordered, let $A$ be a set, and, for all $x$ in $X$, let $A^{X_x}$ be the set of all maps from $X_x$ to $A$. Finally, for all $x$ in $X$, let $r_x:A^{X_x}\to A$ be a map. Then there is a unique map $f:X\to A$ such that $f(x)=r_x(f|X_x)$ for all $x$ in $X$. Here $f|X_x$ denotes the restriction of $f$ to $X_x$. 

\nn\textbf{Lemma.} The map $f$ in the above theorem is unique. 

\nn\emph{Proof.} Let $f$ and $g$ be two maps from $X$ to $A$ satisfying the required conditions. It is easy to see that $\{x\in X\ |\ f(x)=g(x)\}$ is inductive. $\square$ 
%\end{proof} 

\nn\emph{Proof of the Transfinite Recursion Theorem.} Let $\omega$ satisfy $\omega\notin X$, set $Y:=X\cup\{\omega\}$, and decree $\omega>x$ for all $x$ in $X$. Then $Y$ is again well-ordered, and $X=Y_\omega$. We define the subset $T$ of $Y$ as follows: let $y$ be in $Y$. Then $y$ is in $T$ if and only if there is a map $f_y:Y_y\to A$ such that $f_y(x)=r_x(f_y|X_x)$ for all $x$ in $Y_y$. By the Lemma, there is at most one such map $f_y$. 

\nn Claim: $T$ is inductive. Proof. Let $y$ be in $Y$. We can assume that the $f_x$ for $x<y$ have already been defined, and we must define $f_y$, that is, we must define $f_y(x)$ for $x<y$. If $x<z<y$ for some $z$ we set $f_y(x):=f_z(x)$ and the Lemma shows that this does not depend on the choice of $z$. If there is no such $z$, we set $f_y(x):=r_x(f_x)$. It is easy to check that $f_y$ meets our requirements. 

By the Transfinite Recursion Theorem, this implies that $f_\omega$ exists, and it is clear that $f:=f_\omega$ does the job. $\square$ 

\end{document}
